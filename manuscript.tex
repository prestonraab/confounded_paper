\doublespacing


\begin{figure}
	\centering
	\includegraphics[width=\columnwidth]{figures/static/adjuster_workflow.pdf}
	\caption[Batch adjustment justification and steps]{\textbf{Batch adjustment justification and steps.}
	\begin{enumerate*}[(a)]
		\item When measurements are collected from a sample ($X$), systemic effects ($H$) may affect the measurements.
		\item If data from the same sample $X$ is measured under two different conditions, $H_1$ and $H_2$, we may obtain slightly different measurements.
		\item In order to normalize batches of data relative to one another, we first estimate the effect of the hidden variables based on differences in measurements between batches.
		\item Second, we remove the estimated effects in order to normalize the batches relative to one another.
	\end{enumerate*}
	}
	\label{fig:workflow}
\end{figure}


\section{Methods} \label{sec:methods}

An open-source implementation of Confounded is publicly available at \url{https://github.com/jdayton3/Confounded}. Materials for generating this manuscript, including scripts and data, are available at \url{https://github.com/jdayton3/confounded-paper} and \url{https://osf.io/b76ch/}, respectively.
%TODO: We may be able to remove the OSF repo now...but let's wait...

\subsection{Neural network structure}

We developed an adversarial autoencoder network to model and remove confounding effects.
We structured this network in two parts: a variational autoencoder \cite{louizos_variational_2015} to replicate the input (expression) data and a discriminator to detect remaining confounding effects in the autoencoder's output.
By penalizing the autoencoder for the discriminator's success, the autoencoder subnetwork learns over the course of training to output the expression data with confounding effects minimized.
We implemented the neural network in TensorFlow 1.11.0 \cite{abadi_tensorflow_2015} with Python 3.6 \cite{python_software_foundation_python_2019}.
All layers in the network are fully connected, and all activation functions are Rectified Linear Units (ReLUs) \cite{agarap_deep_2018}, except the final layers in the autoencoder and the discriminator, which use the sigmoid function.

\begin{figure}
	\centering
	\includegraphics[width=\columnwidth]{figures/static/network.pdf}
	\caption[Network architecture of Confounded]{\textbf{Network architecture of Confounded.}
	Data with batch effects (represented by different colors) are input into an autoencoder.
	The output of the autoencoder is classified by a discriminator network based on batch.
	The autoencoder is then penalized based on the success of the discriminator.
	Over time, the autoencoder learns to output a faithful representation of the data without signal due to batch.}
	\label{fig:network}
\end{figure}

\subsubsection{Autoencoder}

%% ToDo: This is where we would have to change the default values in the confounded :
%% for DR Piccolo to Change : 
%%
%%
We implemented the variational autoencoder architecture \citep{louizos_variational_2015} described by \citet[Chapter 15]{geron_hands-machine_2017}.
This network has 2 hidden encoding layers and 2 decoding layers, each of size 500. % Is this sample size? I'm not sure where the 500 comes from
The code size is 200. %was: 20.
Each hidden layer is activated with the Exponential Linear Unit (ELU) function \citep{clevert_fast_2015} and trained with the Adam optimizer \citep{kingma_adam_2014} on reconstruction loss (sigmoid cross entropy) combined with latent loss (KL divergence \citep{kullback_information_1951}).

\subsubsection{Discriminator}

We trained the discriminator to predict the original batches based on the autoencoder's output.
The discriminator subnetwork consists of an input layer; four fully connected hidden layers of sizes 1024, 512, 512, and 128, respectively; and an output layer sized based on the number of batches (i.e. if there are 4 distinct batches in the data, the output layer will be size 4).
To combat overfitting and improve training, we also added 50\%-probability dropout \cite{srivastava_dropout_2014} (which prevents overfitting by dropping a random subset of layer inputs in each training iteration) and batch normalization \cite{ioffe_batch_2015} (which helps training by smoothing out the optimization landscape \citep{santurkar_how_2018}) to each layer of the discriminator.
These additions seem to reduce overfitting in the discriminator.

\subsubsection{Loss functions}

To evaluate the network's performance, we used three loss functions.
First, we calculated the autoencoder's loss ($L_A$) by summing the reconstruction loss (sigmoid cross entropy between the autoencoder's input and output) and the latent loss (Kullback-Leibler (KL) divergence \cite{kullback_information_1951} of the code layer).
Second, we calculated the discriminator's loss ($L_D$) as sigmoid cross entropy between its output and a one-hot encoding of the samples' batch labels.
Finally, we trained the autoencoder layers on a combination of the two previous losses,

\begin{equation}
	\label{dual_loss}
	L_{dual} = L_A - \lambda{}(L_D)
\end{equation}

The $\lambda$ value represents a tradeoff parameter for tuning the network's tendency to more faithfully replicate the input or to more completely remove confounding effects.
A higher $\lambda$ value indicates that the network should remove confounding effects more aggressively, whereas a lower value indicates that the network should instead favor faithfully reconstructing the input data.
We did not optimize $L_A$ directly; instead we trained the autoencoder by optimizing $L_{dual}$.

\subsubsection{Training}
%% also another place to change the default values
In all cases, we trained the network using the Adam Optimizer \citep{kingma_adam_2014} with a training rate of 0.0001 for 10,000 iterations on mini-batches of size 256. % was: 100.
In each iteration, we optimized on both $L_D$ and $L_{dual}$.
When optimizing $L_D$ (i.e. training the discriminator), we froze the autoencoder's weights, and vice versa.
We trained the network on a 2017 Dell XPS 15 9560 with an 8-core Intel i7-7700HQ central processing unit and 16 GB of random access memory.
For each dataset, training took roughly 30 minutes to complete, including time to load the input into memory and save the output to disk. % We should probably re-time it to make sure this is still accurate

\begin{figure}
	\centering
	\includegraphics[width=\columnwidth]{figures/final/training_loss.pdf}
	\caption[Autoencoder and discriminator loss]{\textbf{Autoencoder and discriminator loss} over time for one run of Confounded on the MNIST dataset (which we used to validate the network).
	Over the course of training, the autoencoder more faithfully replicates the input data.
	The autoencoder also seems to introduce noise (see the red dashed line near iteration 3100) in response to the discriminator's slight improvements.}
	\label{fig:training_loss}
\end{figure}

\subsection{Comparison to other methods}

We compared Confounded against two other methods that can be used for batch correction. The first was a simple scaling adjuster,
which we implemented in the R programming language (version 3.6.0) \citep{r_core_team_r_2014} using RStudio version 1.2.1194 \citep{rstudio_team_rstudio_2018}.
It adjusts the data by linearly expanding or contracting the values for each variable so that all variables have the same range across the batches.
We also compared against ComBat \citep{johnson_adjusting_2007} using an implementation from the sva package \citep{leek_sva_2017} with some modifications to allow it to work on columns without variance.

We initially intended to test against the SVA \citep{leek_capturing_2007} method but concluded that SVA is more suited for producing surrogate variables for further statistical research rather than removing those variables from the data.
A number of other methods for batch adjustment are available. Some of these use deep neural networks \citep{shaham_removal_2017,shaham_batch_2018,upadhyay_removal_2019}, while others do not\citep{espin-perez_comparison_2018}.
Unfortunately, the datasets we tested did not meet the requirements of these methods (e.g., two equal-sized batches, large sample sizes).

% ----- Brief explanation of the other methods -----
% 1.
% The Shaham lab has two method that we reference up here^, one from 2017 , and one from 2018.
% The 2017 one is the distribution-matching residual net.
% It arbitrarily designates one batch as the "target" and the other batch as the "source".
% It calculates MMD with 3 kernels and averages them, and uses them as the loss function.
% Over training, the source's distribution (according to MMD) should be aligned to the target's distribution.
% In our testing, this was SUPER memory-intensive.
% I think this is the practical reason they reduced their dataset to the first 37 principal components, but their explanation in the paper is that the people who generated the data said the first 37 components contain most of the variation.
% 2.
% The 2018 Shaham paper is about using an adversarial autoencoder to make a batch-free encoding of expression data.
% They have a variational autoencoder with a shared encoder, but separate decoders for each of the 2 batches.
% They say that the decoders output data that has batch info in it but that the encoding layer doesn't have batch info.
% Their encoding layer is only size 15-20, so this is essentially a dimensionality reduction technique--the "batch free encoding" they're producing doesn't have values for each gene.
% 3.
% I don't know how worth-mentioning the Upadhyay paper is, but they basically replicate the DMResNet paper and replace the MMD loss with a discriminator loss like we do.
% They use the Shaham datasets, and I think their code, which is either not publicly available or is just really hard to find, probably has the same caveats as the Shaham code.
% Most particularly, they only account for 2 batches.
% I might have misunderstood their approach though because the English is a little difficult to interpret.


\subsection{Datasets}

We tested each method on four datasets that varied in size and the type of data measured.

\begin{table}
	\centering
	\caption[Dataset information]{
		\textbf{Information} about each dataset used to compare algorithm performance.
	}
	\begin{tabular}{|p{0.13\linewidth}|p{0.14\linewidth}|p{0.09\linewidth}|p{0.14\linewidth}|p{0.18\linewidth}|p{0.14\linewidth}|}
  \hline
  Dataset & Dimensions & Number of Batches & Batch Label & True Class Label & Data Type \\
  \hline
  Bladder Batch & $57\times22,283$ & 5 & Batch & Cancer status & Microarray \\
  \hline
  GSE37199 & $93\times20,024$ & 2 & Plate & Cancer stage & Microarray \\
  \hline
  MNIST & $10,000\times784$ & 2 & Artificial batch & Digit & Grayscale images \\
  \hline
  TCGA Pan-Cancer & $9,366\times325$ & 25 & Cancer Type & TP53 mutation presence & RNA-Seq \\
  \hline
\end{tabular}

	\label{tab:datasets}
\end{table}

%% I keep forgetting whether or not we are leaving the MNIST dataset and just not doing anything, or if we are deleting it out. 
%% maybe put in the simulated expression data and information here 

%\subsection{Simulated Expression}
%We simulated a balanced two class two batch dataset that contains 200 samples of 1000 features each that allowed us to simulate a large dataset to optimize the parameters of the confounded method.
%Initially, we used make_classification from the sklearn classification library \url{}, but combat always outperformed confounded because the clustering made by the make_classification algorithm made it possible for linear models to accurately classify the true classes.
%As we moved away from clustering and added nonlinear patterns to a randomly generated normal distribution set, we saw confounded begin to perform better than combat. 
%When the true class and batch signals are nonlinear and small, combat is not as able to acuratly classify the classes or remove the batches, but confounded is able to identify the classes and effectively remove the batches. 

\subsubsection{MNIST}

Initially, we used an image dataset so we could visually assess how well the batch-correction methods preserve true signal (in this case, the shape and definition of each handwritten digit image).
The MNIST dataset \cite{lecun_mnist_nodate} contains images of handwritten digits that are size-normalized and centered.
We limited our analysis to the 10,000 test images.
We flattened each 28-by-28 image into a 1D vector of size 784 and put each in a CSV file along with the accompanying digit (class) information.
Although convolutional layers are typically used when working with image data, we used fully connected layers to maintain consistency with what would be used on gene-expression data.
However, the autoencoder was still able to represent spatial relationships without explicitly defining them in the model.

\paragraph{Synthetic batch effects}

Because there is no batch information in the MNIST digits dataset, we simulated nonlinear confonding effects.
We applied these effects by iteratively realizing vectors of normally distributed values, multiplying and adding these vectors to the ``expression'' vectors, and applying nonlinearity to the adjusted vector.[@Be more specific in the last part. How did you apply nonlinearity? Or maybe this last part of the sentence is unnecessary.@]
We split the images into two balanced batches (5,000 images for each batch) and included the same number of each digit in each batch.
We applied the same random vectors to each image in a given batch.
Finally, we added random noise to each image to prevent images in a batch from being overly similar to each other.

\subsubsection{Bladderbatch data}

The bladderbatch dataset is a microarray gene-expression dataset from a study of patients with bladder cancer \cite{dyrskjot_gene_2004}.
It is available as an R package \cite{leek_bladderbatch_2017}.
It contains expression values for 57 tissue samples with and without bladder cancer across 5 unbalanced batches.
The dataset has a cancer status (cancerous vs. normal tissue) variable, which we used as the ``true class'' in our validations; it also has a batch variable.
Because bladderbatch is small relative to datasets typically used for deep learning, it helped us to evaluate whether our network overfits on gene-expression datasets of a size typical of applied research.

\subsubsection{Prostate cancer data}

The GSE37199 dataset was profiled using Affymetrix microarrays from patients with advanced castration-resistant prostate cancer \cite{olmos_prognostic_2012}.
We used a curated version of this dataset from \url{http://doi.org/10.17605/OSF.IO/SSK3T}\cite{golightly_curated_2018}.
It contains expression values for 93 tissue samples categorized as either ``Advanced castration resistant'' or ``good prognosis.''
We used these categories for the ``true class'' labels.
This dataset has two types of batch variables---``plate'' and ``centre''---which indicate the microarray plate and research lab that processed the data, respectively.
We adjusted against the ``plate'' variable because it is more granular in nature (with counts of $\{43, 50\}$).

\subsubsection{Pan-cancer data}

The Cancer Genome Atlas (TCGA) produced RNA-Sequencing data for tumors across many cancer types \cite{the_cancer_genome_atlas_research_network_cancer_2013}.
In a previous study \cite{dayton_classifying_2017-1}, we classified this dataset based on the presence or absence of mutations in several known cancer genes.
In that study, we observed that cancer type had a strong confounding effect on our analysis, in which sought to identify patterns spanning many cancer types. We adjusted for this effect using ComBat.
However, we found that a strong nonlinear signal could still be identified by the Random Forests algorithm after adjustment.
Here, we used the same version of TCGA dataset (available at \url{https://osf.io/7xjdn/}),
which has expression values for 9,365 samples across 25 cancer types.
%TODO: Change this to 8744 samples (after averaging across replicates and limiting to patients that had all types of data), 23,386 genes and 24 cancer types?
%TODO: Mention that we used all genes rather than 300?

\subsection{Statistics and metrics}

\subsubsection{Mean squared error}

Mean squared error (MSE) is a measure of how much two vectors or matrices deviate from one another.
It is commonly used to quantify loss when the objective (as with autoencoders) is to minimize the difference between input and output values.

\subsubsection{Maximum mean discrepancy} \label{section:mmd}

In an earlier paper, \citet{shaham_removal_2017} used neural networks to remove batch effects.
%WAS: In a recent paper, \citet{shaham_removal_2017} used neural networks to remove batch effects.
Instead of constraining the autoencoder to remove batch effects based on a discriminator, these researchers trained their network to minimize maximum mean discrepancy (MMD) between batches in an embedded layer of their network.
We calculated MMD using the same formula to determine whether batches appear to come from the same distribution after adjustment.
We used the Gaussian kernel as implemented in \texttt{sklearn.metrics.pairwise.rbf\_kernel} \cite{pedregosa_scikit-learn_2011}.
In cases where there were more than two batches, we averaged all pairwise MMD values to calculate an overall MMD.

\subsubsection{Classification accuracy}

To determine
a) whether batch could still be identified post-adjustment and
b) how well class-related signal was maintained after adjustment,
we used classification algorithms to predict either batch or the ``true class`` labels based on the expression data and used classification accuracy as a metric.
Table \ref{tab:datasets} details which columns were used for these analyses.

%TODO: Update the algorithms listed here.
We used four classification algorithms from the scikit-learn (version 0.19.1) library \cite{pedregosa_scikit-learn_2011}: Random Forests \citep{tin_kam_ho_random_1995}, k-Nearest Neighbors \citep{fix_discriminatory_1951}, and Support Vector Machines (SVM) \citep{cortes_support-vector_1995} with a radial-basis kernel.
%We used four classification algorithms from the scikit-learn (version 0.19.1) library \cite{pedregosa_scikit-learn_2011}: Naive Bayes \citep{maron_automatic_1961}, Random Forests \citep{tin_kam_ho_random_1995}, k-Nearest Neighbors \citep{fix_discriminatory_1951}, and Support Vector Machines (SVM) \citep{cortes_support-vector_1995} with a radial-basis kernel.
In most cases, we used the default hyperparameters, with the exceptions of using Random Forests with \texttt{n\_estimators=10} and using SVM with \texttt{kernel="rbf"}.
[@The defaults in version 0.19.1 of sklearn are n\_estimators=10 for RF and kernel="rbf" for SVM. Did you use these defaults or something different?@] %we used those parameters

We calculated the average classification accuracy across three cross-validation folds, repeated ten times.
%We calculated the average classification accuracy across four cross-validation folds, repeated three times.
We interpreted lower accuracy for batch classification to indicate that batch signal was removed more effectively.
We interpreted higher true-class accuracy to indicate that biologically relevant signal was retained during the batch-adjustment process.
Therefore, given output data from an ideal batch adjuster, batch classification accuracy would be no better than random chance, and true-class accuracy would be equal to what would be obtained for the unadjusted data.

\section{Results} \label{sec:results}

In this study, we created Confounded, an adversarial, variational autoencoder, neural network, to remove nonlinear confounding effects from gene-expression data that may not be accounted for by linear methods.
We compared Confounded to a scaling method and to ComBat \citep{johnson_adjusting_2007} and used qualitative and quantitative techniques to compare the performance of these methods.
Overall, the scaling method performed consistently worse than the other methods;
ComBat and Confounded each excelled in different scenarios, which we describe below.

\subsection{Confounded removes nonlinear confounding effects that other adjusters miss}

[@I also think we should have t-SNE graphs for all three datasets. - Do you think this should be in the main text or just in the supplement? - The supplement.@]

[@Please structure this so that the MNIST results are described first. I think the bladderbatch results should come next because that is the closest thing we have to a reference set for gene expression. Then GSE37199. Then pan-cancer. Please restructure the results and wording to reflect this order. It's OK to mix and match. But the first set of results should be in this order.@]

We used principal component analysis (PCA) and t-distributed Stochastic Neighbor Embedding (t-SNE) \citep{maaten_visualizing_2008} plots to visualize the data before and after batch adjustment.
Before any adjustment, batches (microarray plates) in the GSE37199 dataset showed distinct expression patterns for each batch.
After adjustment with each of the three methods, the batches were less distinct (see \figurename{s} \ref{fig:pca} and \ref{fig:tsne}).
In the PCA plot, Confounded appears to maintain a similar distribution to the unadjusted data, indicating that the underlying distribution has been faithfully reproduced by the networks.
The t-SNE plot shows that the data post-adjustment by Confounded and ComBat appear to cluster less tightly by batch than the unadjusted and scale-adjusted data.
This may indicate an effective removal of nonlinear effects in both cases.
%However, previous research has shown that these plots are not completely trustworthy in representing nonlinear effects \cite{dayton_classifying_2017-1}.

\begin{figure}
	\centering
	\includegraphics[width=\columnwidth]{figures/final/pca.pdf}
	\caption[Principal components analysis (PCA)]{\textbf{Principal components analysis} (PCA) of the GSE37199 dataset before and after batch adjustment with various adjusters.
	None of the datasets appear to be linearly separable.
	Confounded appears to maintain the same distribution of data overall as the unadjusted data while perhaps aligning the batches' distributions.}
	\label{fig:pca}
\end{figure}
\begin{figure}
	\centering
	\includegraphics[width=\columnwidth]{figures/final/tsne.pdf}
	\caption[T-distributed Stochastic Neighbor Embedding (t-SNE)]{\textbf{T-distributed Stochastic Neighbor Embedding (t-SNE)} plot for the GSE37199 dataset before and after adjustment with several algorithms.
	The data seem to cluster less by batch for both Confounded and ComBat, indicating that both adjusters may be removing nonlinear effects in this dataset.}
	\label{fig:tsne}
\end{figure}

%%%To compare Confounded against the other batch adjustment methods, we compared PCA and t-SNE plots along with MSE, MMD, and classifier batch prediction accuracy (using various classifiers).
Confounded shows mixed success with the MSE and MMD metrics.
With MSE, Confounded outperformed the scale adjuster in 3 of the 4 datasets but scored drastically worse on the MNIST dataset, with scores listed in Table \ref{tab:mse} (see also \figurename{} \ref{fig:mse}).
With MMD, Confounded outperformed the scale adjuster again in 3 of the 4 datasets and tied the scale adjuster on the TCGA dataset, with scores listed in Table \ref{tab:mmd} (see also \figurename{} \ref{fig:mmd}).
With both metrics, Confounded consistently performed somewhat worse than ComBat.

\begin{figure}
	\centering
	\includegraphics[width=\columnwidth]{figures/final/mse.pdf}
	\caption[Mean squared error (MSE)]{
		\textbf{Mean squared error} (MSE) between the data prior to and after adjustment with various algorithms.
		Lower MSE represents that the adjuster has more faithfully reproduced the input data.
		MSE for unadjusted data will always be 0 because the input data is identical to the output data.
		Confounded usually performs better than the scale adjuster and somewhat worse than ComBat when measuring MSE.
		See also Table \ref{tab:mse}.
	}
	\label{fig:mse}
\end{figure}
\begin{table}
	\centering
\begin{tabular}{l|l|l|l|l}
\hline
Dataset & Unadjusted & Scale & ComBat & Confounded\\
\hline
\rowcolor{gray!6}  Bladder Batch & 0 & 0.247 & 0.0424 & 0.0698\\
\hline
GSE37199 & 0 & 0.003 & 0.00066 & 0.00168\\
\hline
\rowcolor{gray!6}  MNIST & 0 & 0.00312 & 0.00183 & 0.0187\\
\hline
TCGA & 0 & 9.12e+05 & 1.17e+05 & 1.39e+05\\
\hline
\end{tabular}
	\caption[Mean squared error (MSE)]{
		\textbf{Mean squared error (MSE)} of the unadjusted input data compared to the data output by the given adjusters.
		Lower MSE indicates that the output has changed less from the input.
		See also \figurename{} \ref{fig:mse}.
	}
	\label{tab:mse}
\end{table}
\begin{figure}
	\centering
	\includegraphics[width=\columnwidth]{figures/final/mmd.pdf}
	\caption[Maximum mean discrepancy (MMD)]{
		\textbf{Maximum mean discrepancy} (MMD) between different batches.
		Lower MMD indicates that the distributions of the different batches are more similar.
		In cases with more than two batches, MMD is computed pairwise between each batch and averaged.
		In each case, Confounded usually performs better than the scale adjuster and somewhat worse than ComBat when measuring MSE.
		See also Table \ref{tab:mmd}.
	}
	\label{fig:mmd}
\end{figure}
\begin{table}
	\centering
	\centering
\begin{tabular}{l|l|l|l|l}
\hline
Dataset & Unadjusted & Scale & ComBat & Confounded\\
\hline
\rowcolor{gray!6}  Bladder Batch & 0.48 & 0.463 & 0.183 & 0.258\\
\hline
GSE37199 & 0.0941 & 0.0906 & 0.0255 & 0.0376\\
\hline
\rowcolor{gray!6}  MNIST & 0.113 & 0.0653 & 0.00665 & 0.0117\\
\hline
TCGA & 0.0942 & 0.0942 & 0.0942 & 0.0942\\
\hline
\end{tabular}
	\caption[Maximum mean discrepancy (MMD)]{
		\textbf{Maximum mean discrepancy} (MMD) comparing the distributions of the batches to each other after a given adjustment.
		Lower MMD indicates that the distributions of the different batches are more similar.
		In cases with more than two batches, MMD is computed pairwise between each batch and averaged.
		See also \figurename{} \ref{fig:mmd}.
	}
	\label{tab:mmd}
\end{table}

We would expect that after batch adjustment by an ideal adjuster, batch would no longer be detectable by any machine learning classifier.
Using the batch classification accuracy metric, Confounded seems to outperform other adjusters on larger datasets, whereas ComBat and Confounded seem to perform about the same on smaller datasets (see \figurename{} \ref{fig:batch}).
With both the bladderbatch and GSE37199 datasets, batch classification accuracy decreases well below baseline after batch adjustment with ComBat for all classifiers we tested (see Table \ref{tab:batch}).
%Interestingly, batch accuracy also decreases drastically for the MNIST and TCGA datasets, but only for the Naive Bayes classifier.
%This may be due to two factors: both ComBat and Naive Bayes use Bayesian methods, so ComBat may specifically remove the effects that Naive Bayes identifies; and Naive Bayes does not find patterns based on interactions between variables.
%Although Naive Bayes is no longer able to identify confounding effects in the data after ComBat-adjustment, Random Forests (which does use interactions between variables) still has a very high accuracy for MNIST and an increased accuracy for TCGA.
In contrast, after adjustment by Confounded, the Random Forests algorithm's accuracy decreases more than with any other adjuster for both the MNIST and TCGA datasets.
This indicates that while ComBat's performance may work at least as well as Confounded for smaller expression datasets, Confounded may work better with larger datasets.

\begin{figure}
	\centering
	\includegraphics[width=\columnwidth]{figures/final/batch_accuracy.pdf}
	\caption[Batch classification accuracy]{\textbf{Batch classification accuracy} from 3-fold cross-validation repeated 10 times for several classifiers.
	Lower batch accuracy indicates that more batch-related signal has been removed and therefore indicates better performance.
	Confounded's performance is similar to ComBat's for the smaller datasets and is improved for the larger datasets.
	See also Table \ref{tab:batch}.}
	\label{fig:batch}
\end{figure}
\begin{table}
	\centering
	\centering
\begin{tabular}{l|l|r|r|r}
\hline
Dataset & Adjustment & Baseline & GaussianNB & RandomForest\\
\hline
\rowcolor{gray!6}   & unadjusted &  & 0.626 & 0.688\\
\cline{2-2}
\cline{4-5}
 & scale &  & 0.315 & 0.450\\
\cline{2-2}
\cline{4-5}
\rowcolor{gray!6}   & combat &  & 0.000 & 0.114\\
\cline{2-2}
\cline{4-5}
\multirow[t]{-4}{*}{\raggedright\arraybackslash bladderbatch} & confounded & \multirow[t]{-4}{*}{\raggedleft\arraybackslash 0.333} & 0.386 & 0.315\\
\cline{1-5}
\rowcolor{gray!6}   & unadjusted &  & 0.803 & 0.832\\
\cline{2-2}
\cline{4-5}
 & scale &  & 0.930 & 0.928\\
\cline{2-2}
\cline{4-5}
\rowcolor{gray!6}   & combat &  & 0.238 & 0.423\\
\cline{2-2}
\cline{4-5}
 & confounded &  & 0.338 & 0.353\\
\cline{2-2}
\cline{4-5}
\rowcolor{gray!6}  \multirow[t]{-5}{*}{\raggedright\arraybackslash gse37199} & NA & \multirow[t]{-5}{*}{\raggedleft\arraybackslash 0.538} & 0.662 & 0.670\\
\cline{1-5}
 & unadjusted &  & 0.996 & 1.000\\
\cline{2-2}
\cline{4-5}
\rowcolor{gray!6}   & scale &  & 0.998 & 1.000\\
\cline{2-2}
\cline{4-5}
\multirow[t]{-3}{*}{\raggedright\arraybackslash noisy} & combat & \multirow[t]{-3}{*}{\raggedleft\arraybackslash 0.500} & 0.698 & 1.000\\
\cline{1-5}
\rowcolor{gray!6}   & unadjusted &  & 0.832 & 0.879\\
\cline{2-2}
\cline{4-5}
 & scale &  & 0.846 & 0.965\\
\cline{2-2}
\cline{4-5}
\rowcolor{gray!6}  \multirow[t]{-3}{*}{\raggedright\arraybackslash tcga} & combat & \multirow[t]{-3}{*}{\raggedleft\arraybackslash 0.117} & 0.293 & 0.960\\
\hline
\end{tabular}
	\caption[Batch classification accuracy]{
		\textbf{Batch classification accuracy} for several datasets and adjusters.
		The ideal batch adjuster would completely remove all signal due to batch and would therefore \textit{decrease} batch classification accuracy to around the baseline for all classifiers.
		See also \figurename{} \ref{fig:batch}.
	}
	\label{tab:batch}
\end{table}

With the larger datasets in particular, Confounded outperforms the other adjusters.
On the MNIST dataset, Random Forests is able to detect batch with perfect or near-perfect accuracy after adjustment with the scale adjuster and ComBat, but the highest batch classification accuracy after adjustment by Confounded is Naive Bayes, with an accuracy of 68.8\%.
With TCGA, both the scale adjuster and ComBat drastically increase Random Forests' batch classification accuracy from 95.0\% to 98.5\% and 99.9\% respectively, whereas Confounded decreases the accuracy to 77.8\%.
%With TCGA, both the scale adjuster and ComBat drastically increase Random Forests' batch classification accuracy from 87.6\% to 96.3\% and 97.1\% respectively, whereas Confounded decreases the accuracy to 8.8\%.

\subsection{Class-related signal is still detectable after adjustment by Confounded}

%What is the true class information for Bladderbatch?
With the smaller datasets, Confounded seems to keep true class information roughly as well as ComBat, (with Random Forests, Bladderbatch: 74.3 for ComBat\% and 72.1\% for Confounded, GSE37199: 68.7\% for ComBat and 68.8\% for Confounded; see \figurename{} \ref{fig:true_class} and Table \ref{tab:true_class}).
%With the smaller datasets, Confounded seems to keep true class information roughly as well as ComBat, (with Random Forests, Bladderbatch: 74.3 for ComBat\% and 72.1\% for Confounded, GSE37199: 60.4\% for ComBat and 69.0\% for Confounded; see \figurename{} \ref{fig:true_class} and Table \ref{tab:true_class}).
For the Bladderbatch dataset, true class accuracy is much lower after adjusting with any algorithm, indicating that cancer status and batch may not be independent.

With the larger datasets, Confounded's true class accuracy consistently decreases below the accuracy of other adjusters.
%%This isn't accurate anymore. Our fine tuning made it better. 
%%This may be a good place to include simulated expression results. 
The Confounded's accuracy with Random Forest on our simulated dataset is still much higher than the baseline(70.5\% vs. 50.0\%). %Not sure if that's the right wording
A look at the TCGA data before and after adjustment shows that even with larger datasets, the accuracy of Confounded never decreases below the baseline (72.6\% vs. 69.8\%). 
% compare TCGA datasets here to show that the size of datasets affects the performance. 
%To compare the effect that the size of the data has on Confounded in relation to the other methods, we created two sub-datasets from the TCGA data, a medium and a small and compared their performance. On Random Forest, the ability for machine learning to identify smaples with TP53_Mutated genes as the data size increased changed in Scaled: 71.2 \% to 71.9\% to 80.1\%, in Combat: 71.0\% to 75.2\% to 81.0\%, and Confounded: 67.9\% to 71.0\% to 72.6\%.
%The ability for machine learning classification algorithms to detect the CancerType also increased with increase in size of the dataset. The Random Forest accuracy increased in each adjuster with Scaled: 7.2\% to 89.0\% to 98.5\%, in Combat: 96.3\% to 99.9\% to 99.9\%, and Confounded: 65.8\% to 66.5\% to 78.0\%.
%Though the ability for Confounded to replicate the origional expression data is a little lower than that of the other adjusters, the ability for the Confounded to remove the patterns that allow machine learning to detect CancerType is sgnificantly better than that of the other adjusters which have very high, and sometimes higher accuracy after adjustment than before any kind of adjustment is made. See table? 
%A look at the MNIST digits before and after adjustment (see \figurename{} \ref{fig:mnist}) shows that Confounded's output is often blurry, as is common with the output of variational autoencoders \cite{hou_deep_2016}.
%With MNIST, Confounded's accuracy with Random Forests is still much higher than baseline (84.8\% vs. 11.4\%), but with TCGA, the accuracy decreases below baseline (66.5\% vs. 69.8\%) while the other adjusters' accuracies remain above baseline.
%However, the particular set of parameters that we used in Confounded are likely not optimal for every dataset.
Additional tuning may improve the performance metrics.

\begin{figure}
	\centering
	\includegraphics[width=\columnwidth]{figures/final/true_class_accuracy.pdf}
	\caption[True class classification accuracy]{\textbf{True class classification accuracy} for several datasets and adjusters with 4-fold cross-validation repeated 3 times. A higher accuracy after adjustment is desired because it represents that the adjuster has not destroyed the true class signal.
	See also Table \ref{tab:true_class}.}
	\label{fig:true_class}
\end{figure}
\begin{table}
	\centering
	\centering
\begin{tabular}{l|l|r|r|r}
\hline
Dataset & Adjustment & Baseline & GaussianNB & RandomForest\\
\hline
\rowcolor{gray!6}   & unadjusted &  & 0.908 & 0.859\\
\cline{2-2}
\cline{4-5}
 & scale &  & 0.649 & 0.721\\
\cline{2-2}
\cline{4-5}
\rowcolor{gray!6}   & combat &  & 0.697 & 0.743\\
\cline{2-2}
\cline{4-5}
\multirow[t]{-4}{*}{\raggedright\arraybackslash bladderbatch} & confounded & \multirow[t]{-4}{*}{\raggedleft\arraybackslash 0.702} & 0.698 & 0.721\\
\cline{1-5}
\rowcolor{gray!6}   & unadjusted &  & 0.661 & 0.674\\
\cline{2-2}
\cline{4-5}
 & scale &  & 0.632 & 0.661\\
\cline{2-2}
\cline{4-5}
\rowcolor{gray!6}   & combat &  & 0.647 & 0.647\\
\cline{2-2}
\cline{4-5}
 & confounded &  & 0.605 & 0.690\\
\cline{2-2}
\cline{4-5}
\rowcolor{gray!6}  \multirow[t]{-5}{*}{\raggedright\arraybackslash gse37199} & NA & \multirow[t]{-5}{*}{\raggedleft\arraybackslash 0.667} & 0.605 & 0.669\\
\cline{1-5}
 & unadjusted &  & 0.827 & 0.861\\
\cline{2-2}
\cline{4-5}
\rowcolor{gray!6}   & scale &  & 0.826 & 0.865\\
\cline{2-2}
\cline{4-5}
\multirow[t]{-3}{*}{\raggedright\arraybackslash noisy} & combat & \multirow[t]{-3}{*}{\raggedleft\arraybackslash 0.114} & 0.827 & 0.872\\
\cline{1-5}
\rowcolor{gray!6}   & unadjusted &  & 0.625 & 0.763\\
\cline{2-2}
\cline{4-5}
 & scale &  & 0.603 & 0.758\\
\cline{2-2}
\cline{4-5}
\rowcolor{gray!6}  \multirow[t]{-3}{*}{\raggedright\arraybackslash tcga} & combat & \multirow[t]{-3}{*}{\raggedleft\arraybackslash 0.698} & 0.659 & 0.742\\
\hline
\end{tabular}
	\caption[True class classification accuracy]{
		\textbf{True class classification accuracy} for several datasets and adjusters.
		After adjustment by the ideal batch adjuster, all true class signal should be preserved, and all classifiers should therefore have the same accuracy in predicting true class before and after adjustment.
		See also \figurename{} \ref{fig:true_class}.
	}
	\label{tab:true_class}
\end{table}

\begin{figure}
	\centering
	\includegraphics[width=\columnwidth]{figures/final/mnist.png}
	\caption[MNIST handwritten digits]{\textbf{MNIST handwritten digits}
	\begin{enumerate*}[(a)]
		\item before any adjustment,
		\item with artificial noise added,
		\item adjusted for noise by the scale adjuster,
		\item adjusted for noise by ComBat, and
		\item adjusted for noise by Confounded.
	\end{enumerate*}
	Although Confounded seems to remove more noise from the background, it struggles in some cases to accurately replicate the input data.}
	\label{fig:mnist}
\end{figure}

\section{Discussion} \label{sec:discussion}

\textbf{Why should I use a neural network for batch adjustment?}
The process of measuring data typically leaves confounding effects, especially when the data or measurement process involves living things.
Imagine an experiment where two researchers measure the length of caterpillars.
If one researcher is left-eye dominant and the other is right-eye dominant, they may each see lengths as being a millimeter different.
If both researchers use the same wooden ruler at different relative humidities, the measurements may come out differently due to the wood swelling or due to the caterpillars reactions to humidity.
These same types of problems exist when measuring expression levels, but to more of an extreme---in addition to the target values and measuring equipment changing slightly due to systemic factors, the target values also change in response to each other.
Each of the 20,000 transcript levels may influence or be influenced by other transcript levels.
These cascading network-like effects are extremely likely to have nonlinear components, and our research shows that these nonlinear effects do indeed exist in gene expression data, thus rendering linear batch correction methods insufficient.
Therefore, some form of nonlinear adjustment must be used in order to correct for real-world confounding effects.
%It may just be me, but this senstence is confusing because of the length and the number of commas
Rather than individually model each of infinitely many possible nonlinear interactions to see which represent confounders, we can use a neural network to both approximate and remove the confounders, since neural networks are proven to be universal function approximators \citep{csaji_approximation_2001}.

\textbf{How can I tell how well batch adjustment worked?}
Although the metrics and figures that have been used in the past to validate batch adjustment (such as PCA, MSE, and MMD) represent how well linear effects have been removed, they cannot completely display whether two batches are distinguishable from one another.
Machine learning algorithms are designed specifically to tease out patterns in data that may distinguish one group from another.
We suggest to users of batch correction software that they use machine learning classification accuracy before and after correction in order to determine the degree of batch removal.
We also suggest to researchers in the field of batch correction that classification accuracy be used as a metric in validating their software.
Specifically, the Random Forests algorithm \citep{tin_kam_ho_random_1995} seems to work very well and runs relatively quickly on gene expression data.

\textbf{Which batch adjuster should I use?}
In our testing, ComBat did very well with small ($n < 100$) datasets, even with removing any identifiable nonlinear effects.
However, Confounded outperformed ComBat on the larger datasets according to the batch classification accuracy metric.
In addition to dataset size, researchers selecting a batch adjustment algorithm should consider how important it is for them to accurately replicate their input data.
Such researchers can adjust Confounded's $\lambda$ parameter in order to balance the tradeoff of removing batch and matching the inputs.

\textbf{What limitations does Confounded have?}
\begin{enumerate*}[(a)]
	\item Confounded uses a variational autoencoder, which are known for often outputting a blurry version of the input data (as can be seen in \figurename{} \ref{fig:mnist}).
	However, work in 2016 has identified modifications that may be made to the basic VAE structure to make output images sharper and more realistic \citep{hou_deep_2016}
%	However, recent work has identified modifications that may be made to the basic VAE structure to make output images sharper and more realistic \citep{hou_deep_2016}.
	Similar research with variational autoencoders and gene expression data may yield improved reconstruction losses and decrease the blurring effect.
	\item Confounded takes a long time to run in comparison with ComBat and other linear adjusters.
	Although we acknowledge this as a downfall of many types of machine learning and of neural networks in particular, we believe that 30-60 minutes is a reasonable amount of time for a step that will be run only once per pipeline and that can greatly improve data quality.
	\item It can be difficult to identify the optimal network structure and parameter set for a neural network.
	Though this is the case for many applications of neural networks, we feel that Confounded's default structure worked well in our testing and that it will suffice for most batch correction applications.
	% I wrote that in simulated expression that we optimized the parameters, so we could say that we optimized parameters that worked well across several different datasets and should be good for most applications, but can further be optimized to perform better on individual datasets.
	\item Neural networks usually perform better when given large amounts of data, and traditional batch datasets typically have very few samples.
	We did find that ComBat may outperform Confounded on many smaller, more traditional datasets, but that Confounded performs better on larger datasets.
	However, Confounded did perform reasonably well with the smaller datasets and appeared to avoid overfitting.
	In cases where ComBat is unable to completely remove confounding effects in a small dataset, Confounded may be a viable replacement method.
\end{enumerate*}

\textbf{What else might Confounded be used for?}
At its root, batch correction is a data integration problem:
data from multiple batches must have batch-specific confounding effects removed in order to be treated as one dataset.
Confounded shows promise in removing traditional batch effects from microarray expression data in the Bladderbatch and GSE37199 datasets.
It also effectively decreased artificial batch effects in image data and cancer-type-specific confounding effects in RNA-Seq data.
Confounded may be effective in other data integration problems, such as combining microarray with RNA-Seq datasets, or merging several large datasets measured under different conditions.

Confounded, and adversarial autoencoders in general, show promise as a valuable way to remove confounding biases from expression datasets.
Such methods will enable researchers access to larger datasets, therefore increasing the scope of analyses and furthering science as a whole.

\newpage
\phantomsection
\addcontentsline{toc}{section}{References}
\bibliography{references}

\end{document}
